\section{Démarche}

La problématique initiale telle qu'elle a été énoncée est la suivante : \emph{étant donné un nombre de tatamis, quelles sont les 
configurations possibles ?}\\


Nous allons ici détaillé plus précisément le cahier des charges de l'application. Nous listerons les \emph{epics} afin d'en déduire
les \emph{users stories} et leurs tâches afférentes, et ainsi établir la \emph{roadmap} de notre projet. Le cahier des charges est formulé 
du point de vue de l'utilisateur final, les \emph{epics} étant rédigées sous la forme : \emph{en tant que} \dots ,\emph{je souhaite} \dots ,\emph{afin de } \dots.\\


Par ailleurs afin de prioriser les demandes, nous classerons les fonctionnalités en deux catégories :

\begin{itemize}
    \item essentielles (must have)
    \item optionnelles (nice to have)
\end{itemize}

\section{Fonctionnalités essentielles}

%\item \emph{En tant que } gestionnaire de dojo,\emph{ je souhaite} ... \emph{ afin de }...

\begin{itemize}
    \item \emph{En tant que} gestionnaire de dojo, \emph{ je souhaite} savoir s'il existe une solution utilisant l'ensemble de mes tatamis,
    \emph{afin de } savoir si je pourrai tous les utiliser.
    \item \emph{En tant que} gestionnaire de dojo,\emph{ je souhaite} connaître le nombre de tatamis utilisables \emph{ afin de }
     n'en déployer que le nombre nécessaire.
    \item \emph{En tant que} gestionnaire de dojo  ,\emph{je souhaite} visualiser l'ensemble des dispositions possibles, modulo une rotation ou une symétrie
    \emph{afin de } ne voir sur l'écran que les solutions réellement différentes.
    \item \emph{En tant que} gestionnaire de dojo, \emph{ je souhaite} pouvoir renseigner le nombre de tatamis dont je dispose,
    \emph{afin de } d'obtenir une solution adaptée à mon matériel.
    \item \emph{En tant que} gestionnaire de dojo,\emph{ je souhaite} pouvoir renseigner les dimensions de mon dojo \emph{ afin d'}
     obtenir une solution adaptée à l'espace dont je dispose.    
    \item \emph{En tant que} gestionnaire de dojo,\emph{ je souhaite} voir afficher les dimensions (longueur, largeur, surface) des dispositions 
    proposées  \emph{ afin d' }exploiter aux mieux l'espace disponible à l'intérieur et à l'extérieur du tatamis.
\end{itemize}


\section{Fonctionnalités optionnelles}

\begin{itemize}
    \item \emph{En tant que } gestionnaire de dojo,\emph{ je souhaite} intégrer la possibilité d'utiliser des demi-tatamis 
    \emph{ afin de } remplir au mieux l'espace disponible.
    \item \emph{En tant que } gestionnaire de dojo,\emph{ je souhaite} utiliser un nombre donné de de mi-tatamis 
    \emph{ afin de } pouvoir exploiter tous les demi-tatamis dont je dispose. 
    \item \emph{En tant que } gestionnaire de dojo,\emph{ je souhaite} pouvoir modifier les dimensions d'un tatamis
     \emph{ afin de } obtenir des propositions correspondant à mon matériel.
    \item \emph{En tant que } gestionnaire de dojo,\emph{ je souhaite} pouvoir créer des catégories de couleurs de tatamis
     \emph{ afin de } visualiser des propositions de placement avec les couleurs réelles.
    \item \emph{En tant que } gestionnaire de dojo,\emph{ je souhaite} pouvoir renseigner des critères de placement
     \emph{ afin de } disposer de la meilleure solution selon moi.
\end{itemize}