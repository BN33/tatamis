\section{Choix de méthodologie}

La méthodologie agile nous paraît très adaptée au développement de notre programme.
En effet, la méthodologie agile:

\begin{itemize}
    \item  Est particulièrement adaptée à la résolution de problème complexes
          et incertaines ou l’on ne sait pas forcément avec précisions l’objectif
          final ou la manière d’y arriver, ce qui est notre cas
    \item Est basée sur l'itération avec la production de livrables testables
          à la fin de chaque \emph{sprint}, ce qui semble adapté pour produire nos différentes
          versions (alpha, beta…)
    \item Est basée sur une équipe multidisciplinaire qui couvre toutes les compétences
          pour produire un produit fini et qui s’auto organise, ce qui parait également
          adapté à notre contexte
\end{itemize}

La méthodologie agile, étant particulièrement adaptée aux situations d'incertitude, préconise
une planification au fur et à mesure de temps, plutôt que d’importantes et lourdes activités de 
planification en début de projet car les informations manquent pour cette planification totale en amont.

Pour nous donner un cadre, nous nous inspirons très fortement du schéma “Scrum” et du guide 
Scrum\footnote{The Scrum Guide, Ken Schwaber et Jeff Sutherland, Novembre 2017 https://www.scrum.org/resources/scrum-guide}, 
tout en l’adaptant à notre situation présente avec des ressources limitées et des contraintes particulières.

\section{Les événements}

\subsection{Les sprints et les phases du projet}


Les sprints sont des périodes de développement ayant un objectif précis et permettant d'arriver à une version du programme. 
Compte tenu du planning imposé par l’exercice, les sprints seront de durée variable et d'une durée légèrement supérieur à un 
mois (contrairement à ce qui est suggéré par le schéma Scrum).\\

Les sprints commencent par la session de planning et se terminent par la revue et la rétrospective (événements détaillés ci-après). Elles comprennent également des activités de 'raffinement' ou de préparation du prochain sprint,
 pour qu’un nouveau sprint puisse commencer immédiatement après la clôture du précédent sprint.\\

Quatre sprints sont programmés pour le projet aboutissant aux versions Alpha, Beta, Release Candidate et 
Production.\\

\emph{Nb: Une phase additionnelle de pré-développement aura lieu en amont pour la préparation du projet,
 mais est organisée de manière ad-hoc et ne peut être considérée comme un sprint. 
 Cette phase a pour objectif d’analyser la demande (le cahier des charges), de déterminer la méthodologie
 et gouvernance et de préparer le développement pour aboutir sur une roadmap, un plan de développement 
 global du programme qui sera bien sûr affiné au cours du temps.
}

\subsection{La session de planning du sprint}

Pour chaque sprint la session de planning permet de déterminer:

\begin{itemize}
    \item Le \emph{Quoi}: quels éléments du backlog global peuvent être embarqué dans ce sprint pour créer le Sprint backlog
    \item Le \emph{Comment}: comment chaque élément du Sprint backlog seront techniquement traités
    \item Le \emph{Pourquoi}: quel objectif pour le sprint, sachant que chaque sprint doit délivrer un produit qui peut être limité en fonctionnalités mais qui fonctionne
\end{itemize}

Les décisions sont prises de la manière suivante:
\begin{itemize}
      \item Quoi et pourquoi :
       
      Les propositions d'éléments à ajouter et d’objectif du sprint viennent du product owner. 
      L'équipe de développement prend ensuite la décision de manière souveraine et autonome 
      en session de planification.
       \item Comment :
       
       Chaque élément du Sprint backlog est discuté techniquement pour le décomposer en plus petites 
       tâches qui feront l’objet de tickets.
       
       En cas de manque de compétences techniques, des tickets sont prévus pour la réalisation de recherches.
       
       Cette session couvre également les questions d’architecture du programme: choix du nombre de classes, 
       de leurs interactions…

\end{itemize}

\subsection{"L'hebdo"}

Compte tenu de la situation particulière, un point de contact quotidien comme préconisé par 
le schéma Scrum n’est pas envisageable. Il sera remplacé par:
 \begin{itemize}
       \item Un point hebdomadaire facilité par le Scrum master pour un focus particulier sur l'échange,
        les challenges et solutions. 
       \item La revue des tâches et le statut sont quand à eux discutés à travers:
       \begin{itemize}
             \item Une communication continue sur Slack
             \item Une mise à jour en direct des avancées sur Redmine, directement sur les tâches. 
             Redmine apportant la visibilité nécessaire à tous les membres de l'équipe pour comprendre 
             le statut rapidement grâce à la combinaison du diagramme Gantt, d’un tableau Kanban 
             des tâches et d’un tableau de roadmap qui suit le pourcentage de complétude du backlog 
             du sprint

       \end{itemize}
 \end{itemize}


\subsection{La revue du sprint}

Chaque sprint se termine par une revue du produit livré à la fin du sprint. Les fonctionnalités 
développées sont discutées, ainsi que les challenges rencontrés.
L'équipe commence à se projeter également sur le prochain sprint et ce qu’il faut faire ensuite.

\subsection{La retrospective}
L’objectif de cette réunion est une introspection pour une amélioration continue notamment de 
la collaboration au sein de l'équipe, les processus et les outils.
La session est facilitée par le Scrum master et se déroule selon les principes suivants:

\begin{itemize}
      \item Discussions autour de 3 blocs successivement:
      \begin{itemize}
            \item \emph{Garde}: ce que l’on considère adapté et à conserver dans le futur
            \item \emph{Start}: ce que l’on souhaite commencer à faire pour améliorer la situation
            \item \emph{Stop}: ce que l’on souhaite arrêter sur un constat d'échec
      \end{itemize}
      \item Étapes de chaque blocs:
      \begin{itemize}
            \item Réflexion individuelle de points/idées à ajouter pour le bloc discute
            \item Discussion de groupe pour regrouper les points mentionnés par thème
            \item Résumé des actions/idées retenues
      \end{itemize}
      \item La session se termine par la détermination du plan d'amélioration regroupant les 
      idées retenues et en ajoutant une composante de temps et responsabilité des actions
      \item Outil utilisé pour la session: Miro (outil interactif et collaboratif permettant 
      notamment de créer et arranger facilement des "post it" représentant les points/idées)
\end{itemize}

\section{Les éléments de formalisation}

\subsection{Cahier des charges global (et les Epics)}
Le cahier des charges global représente la demande et le besoin de l’utilisateur. 
Il comprend une description du/des type d’utilisateur(s). Il est constitué des fonctionnalités majeures 
que l’utilisateur souhaite pouvoir effectuer grâce au programme. 

Il est constitué de quelques “Epics” qui expliquent le type d’utilisateur, l’objectif et la raison. 
Les epics sont rédigées sous la forme suivante:

“En tant que …, je souhaite …, afin que …” 

Les épics ont les statuts suivants Statut: Backlog (non planifié), À commencer, En cours, Achevée, Rejeté.\\



\noindent%
\hfill%
\begin{minipage}{12cm}
      \textsl{Définition d'achèvement ('Definition of Done') des epics :}

      \textsl{Pour assurer une transparence et que toutes les parties prenantes aient la même définition, 
une définition d'achèvement est formalisée ainsi:}
\begin{enumerate}     
      \item  \textsl{Toutes les users stories de l' epic sont achevés}
      \item  \textsl{Les activités de refactoring ont été réalisées (pour vérification de la concordance 
      avec les principes de développement SOLID)}
      \item  \textsl{Les tests utilisateurs sont réalisés et leur résultat est positif}
\end{enumerate}
\end{minipage}

\subsection{Le backlog global (et les User Stories)}

Le backlog est constituée de l’ensemble d'éléments qui pourraient être construits (codés) pour arriver 
à un produit fini idéal. Il représente des besoins très spécifiques des utilisateurs. 

Il est constitué de \emph{User stories} (scénarios utilisateur) qui expliquent la même chose et 
sont rédigées de la même manière que les epics, mais qui représentent de plus petits éléments.

Une \emph{User story} représente une fonctionnalité très particulière et sa structure est la suivante:

\begin{itemize}
      \item Nom de la fonctionnalité
      \item Contexte
      \item Utilisateur (“En tant que”), Objectif (“je souhaite"), Raison (“afin que”)
      \item Test utilisateur d’acceptance (plus d’explication à la suite)
      \item Statut: Backlog (non planifié), À commencer, En cours, Achevée, Rejeté
      \item Sprint de rattachement (ou Backlog global si pas encore planifié au sein d’un sprint)
      \item Priorisation (proposition de sprint)
      \item Documentation utilisateur
      \item Vérification de complétude (voir page \pageref{DefofDoneUS} la définition d'achèvement)
\end{itemize}

\bigskip

Les user stories suivent le principe \emph{INVEST}:
\begin{itemize}
      \item \emph{Independant} (Indépendante) : chaque user story est indépendante des autres
      \item \emph{Negotiable} (Négociable) : une user story décrit un besoin; la manière d’y répondre reste 
      négociable
      \item \emph{Valuable} (Apporte de la valeur) : chaque user story doit apporter de la valeur à l'utilisateur
      \item \emph{Estimable} (Estimable) : une user story doit être estimable par l'équipe de développement, 
      c’est a dire que l'équipe de développement doit avoir assez d’information pour comprendre l’effort nécessaire à la mise en oeuvre
      \item \emph{Small} (Petit) : une user story doit être petite, c’est à dire traitable en quelques jours 
      (difference avec un Epic)
      \item \emph{Testable} (Testable) : une user story doit pouvoir être testable, ce qui permet de s’assurer 
      qu’elle est assez bien définie. 
\end{itemize}

Pour assurer la cohérence avec le modèle INVEST, et notamment la valeur, l’estimabilité et la testabilité, 
les tests unitaires d’acceptance sont décrits pour chaque user story et sont notamment un des paramètres 
de la définition d'achèvement.\\

\noindent%
\hfill%
\begin{minipage}{12cm}
      \textsl{\label{DefofDoneUS}Définition d'achèvement ('Definition of Done') des user stories :}

      \textsl{Pour assurer une transparence et que toutes les parties prenantes aient la même définition, 
une définition d'achèvement est formalisée ainsi:}
\begin{enumerate}     
      \item  \textsl{Toutes les tâches des users stories sont achevés}
      \item  \textsl{Les test utilisateur d’acceptance sont réalisés et leur résultat est positif 
      (en cas de bugs, ils ont étés corrigés)}
      \item  \textsl{La documentation utilisateur est terminée}\\
\end{enumerate}
\end{minipage}


Les User stories peuvent être créées à tout moment du projet. Elles commencent toujours par être ajoutées 
au Backlog global, notamment lors des activités de 'raffinement' en préparation du prochain sprint. 
Elles sont ensuite ajoutées au Backlog d’un sprint particulier en session de planification. 
Une partie importante des User stories sera créer en phase de pré-développement, mais de nombreuses 
User stories seront également créer au fur et à mesure du temps et que le développement avance.

Le backlog global est ordonné par le Product Owner, c'est-à- dire que le Product Owner propose 
les tâches à embarquer pour chaque sprint en session de planification. 


\subsection{Roadmap globale}

La 'roadmap' (feuille de route) globale est un plan de développement du programme, pour arriver au 
produit fini. Elle reprend les Epics et User stories de chacune des étapes pour arriver au produit final:
\begin{itemize}
      \item Pre-développement
      \item Versions successives: Alpha, Beta, Release Candidate, Production
\end{itemize}

La roadmap peut également en vue très détaillée reprendre les tâches de chaque sprint.

Comme expliqué plus haut, en méthodologie agile, la roadmap est vivante et se construit au fur et à mesure 
du temps, comme expliqué plus haut. Un premier jet en lance lors de la phase de pré-développement 
mais c’est un travail continue et la roadmap est affinée en permanence, notamment par l’ajout de User stories 
ou de tâches permettant d’arriver à l'objectif.

L’objectif est d'éviter les tâches de planification trop lourdes et trop incertaines en amont, 
alors que l’incertitude est forte, et de planifier plutôt au fil du temps dans des conditions de 
meilleure connaissance.

Ainsi, plusieurs versions de la roadmap seront présentées au cours du projet. 



\subsection{Le backlog d’un sprint (et les Tâches)}

Le backlog d’un sprint est le plan de développement d’un sprint et est le résultat des discussions 
de la session de planification. Il comprend:
\begin{itemize}
      \item La liste des user stories qui seront traitées dans le sprint
      \item La décomposition des user story en une liste des tâches par user story pour la réaliser
\end{itemize}

Une tâche est un petit élément de code, une pièce du puzzle pour arriver à réaliser la user story 
dans son ensemble.
Toute tâche comprend une documentation technique et une définition des tests d’acceptance unitaire, 
pour permettre au développeur de bien comprendre le résultat attendu. 

\noindent%
\hfill%
\begin{minipage}{12cm}
      \textsl{Définition d'achèvement ('Definition of Done') des tâches :}

      \textsl{Pour assurer une transparence et que toutes les parties prenantes aient la même définition, 
une définition d'achèvement est formalisée ainsi:}
\begin{enumerate}     
      \item \textsl{Le code est écrit}
      \item \textsl{Le code a été revu par un autre développeur}
      \item \textsl{Les test unitaires sont réalisés et leur résultat est positif (en cas de bugs, 
      ils ont étés corrigés)}
      \item \textsl{La documentation utilisateur est terminée}
      \item \textsl{La documentation technique a été revue par un autre développeur}
\end{enumerate}
\end{minipage}

\section{Les rôles}

\subsection{Description des rôles}

\begin{itemize}
      \item \textbf{Le 'product owner'}
      
      Il est responsable de maximiser la valeur créée par le programme résultant du travail de l'équipe. 
      Ses tâches principales consistent en:
      \begin{itemize}
            \item Exprimer les éléments du Backlog
            \item Ordonner le Backlog par ordre de priorité
            \item Assurer la clarté du backlog et que ces éléments soient bien compris de tous
      \end{itemize}

      \item \textbf{Les membres de l'équipe de développement}
      
      Ils sont responsables de la création du produit à l'issue de chaque sprint. 
      Ils s’auto-organisent et n’ont pas de titre ou hiérarchie au sein de l'équipe

      \item \textbf{Le 'scrum master'}
      
      Il aide et supporte l'utilisation de la méthodologie scrum. Ses tâches principales consistent en:
      \begin{itemize}
            \item Aider le product owner dans la gestion du backlog, notamment avec les techniques adaptées
            \item Coache les membres de l'équipe de développement pour s’auto-organiser et intervient en cas de blocage
            \item Interagit avec le reste de l’organisation (par exemple le Big Boss et le Directeur Technique) pour que 
            l'équipe conserve son focus sur le développement
      \end{itemize}

\end{itemize}

\subsection{Assignation des rôles}

Compte tenu de la taille réduite de l'équipe, les membres peuvent être amenés à jouer plusieurs rôles. 
L’organisation de l'équipe sera la suivante:\\


\begin{tabular}{|l|p{8cm}|l|}
      \hline
      \rowcolor{orange} Titre & Rôle au sein de l'équipe Scrum & Nom \\
      \hline
      Big Boss & & Tristan COLOMBO \\
      \hline
      Directeur Technique & & Tristan COLOMBO \\
      \hline
      Chef de projet & Membre de l'équipe de développement & Bruno BOURGINE \\
      \hline
      Développeur & Scrum master et Product Owner et membre de l'équipe de développement & Benoit ANGER \\
      \hline
      Développeur & Membre de l'équipe de développement & Ismail BOULALEH \\    
      \hline  
\end{tabular}

\section{L'approche de test}

L’approche de test sera à niveaux multiples:
\begin{itemize}
      \item Test unitaires pour les tâches
      \item Test utilisateurs pour les Epics et User stories
\end{itemize}


Dans tous les cas, les tests sont définis en amont, au moment de la rédaction des epics, user stories et tâches.
L’objectif est de:

\begin{enumerate}
      \item permettre au développeur de bien comprendre le résultat attendu tout en lui laissant la liberté 
      pour l’atteindre et 
      \item d’apporter une transparence à toutes les parties prenantes.
\end{enumerate}

 
