\section{Outils de communications}

\subsection{Redmine}

Cet outil de gestion de projet a été choisi en premier lieu pour sa disponibilité immédiate (il n'y a pas eu d'installation
ou de paramétrage de serveur à réaliser) mais aussi pour sa complétude en terme d'outils. Il dispose
en effet de l'ensemble des fonctionnalités dont nous avions besoin pour ce qui est de la création, de l'ordonnancement
et du suivi des demandes. En cela il est tout à fait adapté à la méthodologie choisie.\\

Nous avons pu le paramétrer un peu plus finement de façon à ce que les caractéristiques et l'évolution des demandes
correspondent à la terminologie employée pour détailler notre projet : type de tracker, statut des demandes, champs personnalisés\dots


\subsection{Slack}

Slack a été choisi comme outil de communication entre les membres de l'équipe afin d'établir des canaux de discussion différenciés. 
Cela permet à l'équipe des échanges plus ciblés et donc plus efficaces, ainsi qu'une vision plus ordonnée de l'historique des communications.

\subsection{Github}

La plateforme Github a été choisie pour héberger et gérer l'ensemble des éléments du projet, à savoir
le code de l'application ainsi que le rapport.

Le dépot est consultable à l'adresse : \url{https://github.com/bubobou/tatamis}

\section{Développement}

\subsection{Langage}

De part sa facilité d'assimilation et les nombreuses bibliothèques disponibles, le choix du langage de programmation
s'est porté sur Python dans sa version 3.9.

\subsection{Bibliothèques}

Différentes bibliothèques nous ont parues d'emblée utiles pour aborder ce projet. Tout d'abord
une recherche documentaire nous a menée vers la bibliothèque \textsf{facile} permettant de traiter
de la programmation par contrainte. Même s'il ne sera pas forcément retenue dans la version finale, 
sa disponibilité nous permet d'aborder plus sereinement notre problématique.\\

Par ailleurs dans la finalité d'une application avec interface graphique, potentiellement portée sur terminal mobile,
 nous avons envisagé l'utilisation de bibliothèques et d'utilitaires tels que :

 \begin{itemize}
     \item \textsf{PyQt}
     \item \textsf{Kivy}
     \item \textsf{python for android}
 \end{itemize}