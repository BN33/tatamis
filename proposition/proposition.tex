\documentclass{scrartcl}
\usepackage[utf8]{inputenc}
\usepackage[T1]{fontenc}
\usepackage[frenchb]{babel}
\usepackage{lmodern}
\usepackage{geometry}
\usepackage{enumitem, pifont}
\usepackage{xcolor}
\usepackage{amsfonts}
\usepackage{amssymb}
\usepackage{amsmath}
\usepackage[cache=false]{minted}



\setlist[itemize, 1]{label = {\textbullet},}

\geometry{top=2cm, bottom=2cm, left=2cm, right=2cm}

\begin{document}
    \title{Projet maths-info : proposition}
    \subtitle{Pavage avec des tatamis}
    \author{Bruno Bourgine\\}
    \maketitle

    \section{Descriptif de la problématique}

    Le pavage du plan avec des rectangle est un problème classique et déjà largement documenté,
    mais je souhaite l'aborder par un aspect très concret : étant donné un nombre de tatamis, quels sont les 
    configurations possibles.
    
    C'est un problème que rencontre notamment toute personne qui se retrouve à devoir installer un dojo.
    Il existe une contrainte de base qui est que 4 tatamis ne rejoignent jamais en un même coin. Mais ont peut
    en ajouter d'autres : possibilité de demi-tatamis (carré), répartition des couleurs, répartition de l'usure...

    \section{Cahier des charges}

    Utilisateur final : gestionnaire de dojo
    Terminal : a priori multi-plateforme, mais de préférence mobile

    \section{Solutions existantes}

    \section{Langages et outils envisagés}

\end{document}
