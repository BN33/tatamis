\section{Démarche}

La problématique initiale telle qu'elle a été énoncée est la suivante : \emph{étant donné un nombre de tatamis, quelles sont les 
configurations possibles ?}\\

Les dojos sont de dimension m (hauteur) x n (largeur) unités. Et les tatamis sont de dimension 1 x 2 unités. Il existe parfois des demi-tatamis de dimension 1 x 1 unité.
Pour disposer les tatamis, il existe une contrainte: quatre coins de tatamis ne peuvent pas se retrouver en un même point.

Nous allons ici détaillé plus précisément le cahier des charges de l'application. Nous listerons les \emph{epics} afin d'en déduire
les \emph{users stories} et leurs tâches afférentes, et ainsi établir la \emph{roadmap} de notre projet. Le cahier des charges est formulé 
du point de vue de l'utilisateur final, les \emph{epics} étant rédigées sous la forme : \emph{en tant que} \dots ,\emph{je souhaite} \dots ,\emph{afin de } \dots.\\


Par ailleurs afin de prioriser les demandes, nous classerons les fonctionnalités en deux catégories :

\begin{itemize}
    \item essentielles (must have)
    \item optionnelles (nice to have)
\end{itemize}

\section{Persona de l’utilisateur final}

L’utilisateur final est un gestionnaire de dojo.\\

A propos:
\begin{itemize}
    \item Les gestionnaires de dojo ont des profils et backgrounds variés. Il peut être très avancé 
    ou très novice en informatique et en géométrie
    \item En revanche, avec un peu de formation et si on lui installe les outils, il sera capable 
    des les utiliser même si l’ergonomie n’est pas idéale
\end{itemize}


Objectifs:
\begin{itemize}
    \item Préparer le dojo pour qu’il soit prêt pour les entraînements et compétitions
    \item Assurer l’usure équitable des tatamis par des modifications régulières de disposition
\end{itemize}

Points de douleur:
\begin{itemize}
    \item Quand il arrive sur un nouveau dojo, il est difficile de savoir rapidement si une 
    combinaison de tatami est possible ou non
    \item Il est également difficile de savoir combien de tatamis lui sont nécessaires pour 
    faire le dojo
    \item A chaque fois qu’il faut nettoyer le dojo, il faut enlever tous les tatamis, et 
    il est compliqué de refaire le dojo
    \item En particulier lorsqu’il a des demi-tatamis ou des tatamis de couleurs différentes

\end{itemize}



\section{Fonctionnalités essentielles}

%\item \emph{En tant que } gestionnaire de dojo,\emph{ je souhaite} ... \emph{ afin de }...

\begin{itemize}
    \item En tant que gestionnaire de dojo, je souhaite savoir s’il existe une solution pour
     un dojo d'une dimension donnée, afin de savoir si je pourrais remplir mon dojo ou non.
\end{itemize}

\textbf{ Si une solution existe :}

\begin{itemize}
    \item \emph{En tant que} gestionnaire de dojo, 
    \emph{ je souhaite} connaître le nombre de tatamis nécessaires pour un dojo d'une dimension donnée,
    \emph{afin de } n’en déployer que le nombre nécessaire.
    \item \emph{En tant que} gestionnaire de dojo,
    \emph{ je souhaite} de dispositions possibles pour un dojo d'une dimension donnée, 
    \emph{ afin de } ne voir sur l’écran que les solutions réellement différentes.
    \item \emph{En tant que} gestionnaire de dojo ,
    \emph{ je souhaite} visualiser l'ensemble des dispositions possibles, pour un dojo d'une dimension donnée
    \emph{afin de } m’aider à placer les tatamis sur mon dojo.
    \item \emph{En tant que} gestionnaire de dojo, 
    \emph{ je souhaite} pouvoir renseigner le nombre de tatamis dont je dispose, 
    \emph{afin de } d'obtenir une solution adaptée à mon matériel.
    \item \emph{En tant que} gestionnaire de dojo,
    \emph{ je souhaite} voir afficher les dimensions (longueur, largeur, surface) des dispositions proposées  
    \emph{ afin d' }exploiter aux mieux l'espace disponible à l'intérieur et à l'extérieur du tatamis.
\end{itemize}


\section{Fonctionnalités optionnelles}

\textbf{ Si une solution existe :}
\begin{itemize}
    \item \emph{En tant que} gestionnaire de dojo, 
    \emph{ je souhaite} connaître le nombre de dispositions possibles, modulo une rotation ou une symétrie 
    \emph{ afin de } ne voir sur l’écran que les solutions réellement différentes.
    \item \emph{En tant que} gestionnaire de dojo, 
    \emph{ je souhaite} visualiser l’ensemble des dispositions possibles, modulo une rotation ou une symétrie 
    \emph{ afin de } ne voir sur l’écran que les solutions réellement différentes.
\end{itemize}

\textbf{Peu importe si une solution existe sans demi-tatamis:}
\begin{itemize}
    \item \emph{En tant que } gestionnaire de dojo,
    \emph{ je souhaite} je souhaite savoir s'il est utile d’utiliser des demi-tatamis
    \emph{ afin de } trouver une solution quand elle n’existe pas sans demi-tatamis.
    \item \emph{En tant que } gestionnaire de dojo,
    \emph{ je souhaite} avoir visuellement les dispositions possibles en utilisant un nombre donné de demi-tatamis
    \emph{ afin de } pouvoir exploiter tous les demi-tatamis dont je dispose. 
    \item \emph{En tant que } gestionnaire de dojo,
    \emph{ je souhaite} pouvoir modifier les dimensions d'un tatamis
     \emph{ afin de } obtenir des propositions correspondant à mon matériel.
    \item \emph{En tant que } gestionnaire de dojo,
    \emph{ je souhaite} pouvoir créer des catégories de couleurs de tatamis
     \emph{ afin de } visualiser des propositions de placement avec les couleurs réelles.
    \item \emph{En tant que } gestionnaire de dojo,
    \emph{ je souhaite} pouvoir renseigner des critères de placement (par exemple une symétrie, un motif particulier…) 
    \emph{ afin de } disposer de la meilleure solution selon moi.
\end{itemize}