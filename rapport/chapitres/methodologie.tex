\section{Choix de méthodologie}

La méthodologie agile nous paraît très adaptée au développement de notre programme.
En effet, la méthodologie agile:

\begin{itemize}
    \item  Est particulièrement adaptée à la résolution de problème complexes
          et incertaines ou l’on ne sait pas forcément avec précisions l’objectif
          final ou la manière d’y arriver, ce qui est notre cas
    \item Est basée sur l'itération avec la production de livrables testables
          à la fin de chaque \emph{sprint}, ce qui semble adapté pour produire nos différentes
          versions (alpha, beta…)
    \item Est basée sur une équipe multidisciplinaire qui couvre toutes les compétences
          pour produire un produit fini et qui s’auto organise, ce qui parait également
          adapté à notre contexte
\end{itemize}

La méthodologie agile, étant particulièrement adaptée aux situations d'incertitude, préconise
une planification au fur et à mesure de temps, plutôt que d’importantes et lourdes activités de planification en début de projet car les informations manquent pour cette planification totale en amont.

Pour nous donner un cadre, nous nous inspirons très fortement du schéma “Scrum” et du guide Scrum1, tout en l’adaptant à notre situation présente avec des ressources limitées et des contraintes particulières.

\section{Les événements}

\subsection{Les sprints et les phases du projet}


Les sprints sont des périodes de développement ayant un objectif précis et permettant d'arriver à une version du programme. 
Compte tenu du planning imposé par l’exercice, les sprints seront de durée variable et d'une durée légèrement supérieur à un mois (contrairement à ce qui est suggéré par le schéma Scrum).
Les sprints commencent par la session de planning et se terminent par la revue et la rétrospective (événements détaillés ci-après). Elles comprennent également des activités de 'raffinement' ou de préparation du prochain sprint, pour qu’un nouveau sprint puisse commencer immédiatement après la clôture du précédent sprint.
Quatre sprints sont programmés pour le projet aboutissant aux versions Alpha, Beta, Release Candidate et Production.

Nb: Une phase additionnelle de pré-développement aura lieu en amont pour la préparation du projet, mais est organisée de manière ad-hoc et ne peut être considérée comme un sprint. Cette phase a pour objectif d’analyser la demande (le cahier des charges), de déterminer la méthodologie et gouvernance et de préparer le développement pour aboutir sur une roadmap, un plan de développement global du programme qui sera bien sûr affiné au cours du temps.

\subsection{La session de planning du sprint}

Pour chaque sprint la session de planning permet de déterminer:

\begin{itemize}
    \item Le ‘Quoi’: quels éléments du backlog global peuvent être embarqué dans ce sprint pour créer le Sprint backlog
    \item Le ‘Comment’: comment chaque élément du Sprint backlog seront techniquement traités
    \item Le ‘Pourquoi’: quel objectif pour le sprint, sachant que chaque sprint doit délivrer un produit qui peut être limité en fonctionnalités mais qui fonctionne
\end{itemize}

Les décisions sont prises de la manière suivante:
 Quoi et pourquoi
Les propositions d'éléments à ajouter et d’objectif du sprint viennent du product owner. L'équipe de développement prend ensuite la décision de manière souveraine et autonome en session de planification.